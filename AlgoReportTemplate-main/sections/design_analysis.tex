Los algoritmos de \textbf{fuerza bruta} y \textbf{programación dinámica} utilizarán las siguientes funciones para calcular los costos, dependiendo de la letra involucrada. Según el tipo de operación, se consultará el archivo correspondiente. En el caso de \textbf{inserción} o \textbf{eliminación}, se accederá a archivos que contienen una fila de 26 números, cada uno representando el costo de operación para una letra específica, determinado por su índice. Para las operaciones de \textbf{sustitución} o \textbf{transposición}, se empleará una matriz simétrica de 26x26, en la cual se registra el costo de cambiar una letra por otra. 

\begin{algorithm}[H]
    \SetKwFunction{CostoInsert}{costo\_insert}
    \SetKwFunction{CostoDelete}{costo\_delete}
    \SetKwFunction{CostoSub}{costo\_sub}
    \SetKwFunction{CostoTranspose}{costo\_transpose}
    
    \DontPrintSemicolon
    \footnotesize
    
    \SetKwProg{myproc}{Function}{}{}
    
    % Función para la operación de inserción
    \myproc{\CostoInsert{b}}{
        \Return costosInsert[b - 'a']\;
    }
    
    % Función para la operación de eliminación
    \myproc{\CostoDelete{a}}{
        \Return costosDelete[a - 'a']\;
    }
    
    % Función para la operación de sustitución
    \myproc{\CostoSub{a, b}}{
        \Return costosReplace[a - 'a'][b - 'a']\;
    }
    
    % Función para la operación de transposición
    \myproc{\CostoTranspose{a, b}}{
        \Return costosTranspose[a - 'a'][b - 'a']\;
    }
    
    \caption{Funciones de costos para las operaciones de edición}
    \label{alg:funcionesCostos}
\end{algorithm}


Se utilizó como guía los algoritmos y las explicaciones presentadas en Wikipedia sobre la Distancia de Levenshtein ~\cite{eswiki:157799255}, que busca el número mínimo de operaciones requeridas para transformar una cadena de caracteres en otra, y el algoritmo Smith-Waterman ~\cite{eswiki:160948078}, una estrategia para realizar alineamiento local de secuencias basado en programación dinámica. También se utilizó la explicación en GeeksforGeeks sobre la distancia de Levenshtein ~\cite{zaidkhan15_levenshtein} para la confección de los algoritmos necesarios para esta experiencia.

\epigraph{\textit{``Brute-force algorithm, which is also called the “naïve” is the simplest algorithm that can be used inpattern searching. It is probably the first algorithm we might think of for solving the pattern searching problem. It requires no preprocessing of the pattern or the text''}} {--- \citeauthor{mohammad2006occurrences}, \citeyear{mohammad2006occurrences} \cite{mohammad2006occurrences}}



% Autor
% Luis Zegarra Stuardo
% 202073628-6
% Tarea 2 y 3
% Algoritmos y Complejidad 2024-2