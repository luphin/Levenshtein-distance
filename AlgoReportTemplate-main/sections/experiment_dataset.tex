\begin{mdframed}
    \textbf{La extensión máxima para esta sección es de 2 páginas.}
\end{mdframed}

Para los casos de prueba, se decidió no generarlos de manera automática, con el objetivo de mantener control sobre los casos a evaluar y verificar que los algoritmos ejecuten correctamente los pasos necesarios para alcanzar la solución. Los casos de prueba fueron diseñados tomando en cuenta los siguientes escenarios:

\begin{itemize} 
    \item Casos donde las cadenas están vacías: Ejemplos incluyen S1 = $""$ y S2 = $""$, S1 = $"$aaa$"$ y S2 = $""$, \\ 
    y S1 = $""$ y S2 = $"$xyz$"$. 
    \item Casos con caracteres repetidos: Ejemplos incluyen S1 = $"$aabb$"$ y S2 = $"$ab$"$, y S1 = $"$abc$"$ y \\
    S2 = $"$abc$"$. 
    \item Casos donde las transposiciones son necesarias: Ejemplos incluyen S1 = $"$abcd$"$ y S2 = $"$abdc$"$, y \\
    S1 = $"$ab$"$ y S2 = $"$ba$"$. 
    \item Casos variando el largo de las cadenas.
\end{itemize}

Los casos de prueba están almacenados en el archivo \texttt{casos\_prueba.txt}. En este archivo se pueden agregar las líneas que se deseen probar, utilizando el formato ``\texttt{palabra1 palabra2}``. Para representar una cadena vacía, se debe emplear el símbolo de comillas dobles ($""$).

\subsection{Documentación de Casos de Prueba}

A continuación se presentan los casos de prueba utilizados (Figura \ref{fig:teststrings}) para evaluar los algoritmos de distancia de edición. En cada uno se especifican las entradas \( S1 \) y \( S2 \), las operaciones esperadas, el costo total estimado y la justificación de la salida.

\begin{table}[!ht]
    \centering
    \footnotesize
    \begin{tabular}{|c|l|l|p{5cm}|p{4cm}|}
    \hline
    \textbf{Caso de Prueba} & \textbf{Entrada \( S1 \)} & \textbf{Entrada \( S2 \)} & \textbf{Operaciones Esperadas} & \textbf{Costo Total Esperado} \\
    \hline
    1 & $"$abc$"$ & $"$abc$"$ & Ninguna operación. & 0 \\
    2 & $"$ab$"$ & $"$ba$"$ & Transposición de 'a' y 'b'. & Costo transposición \\
    3 & $"$abc$"$ & $"$acb$"$ & Transposición de 'b' y 'c'. & Costo transposición \\
    4 & $"$abcde$"$ & $"$abcde$"$ & Ninguna operación. & 0 \\
    5 & $"$abc$"$ & $"$a$"$ & Eliminación de 'b', Eliminación de 'c'. & 2 * Costo eliminación \\
    6 & $"$abc$"$ & $"$def$"$ & Sustitución de 'a' $\rightarrow$ 'd', 'b' $\rightarrow$ 'e', 'c' $\rightarrow$ 'f'. & 3 * Costo sustitución \\
    7 & $"$abcd$"$ & $"$abdc$"$ & Transposición de 'c' y 'd'. & Costo transposición \\
    8 & $"$aaa$"$ & $"$""$"$ & Eliminación de 'a', 'a', 'a'. & 3 * Costo eliminación \\
    9 & $"$""$"$ & $"$xyz$"$ & Inserción de 'x', 'y', 'z'. & 3 * Costo inserción \\
    10 & $"$""$"$ & $"$""$"$ & Ninguna operación. & 0 \\
    11 & $"$cuadrado$"$ & $"$cuaresma$"$ & Sustitución de 'd' $\rightarrow$ 'r', 'r' $\rightarrow$ 'e', 'a' $\rightarrow$ 's', 'd' $\rightarrow$ 'm', 'o' $\rightarrow$ 'a'. & 5 * Costo sustitución \\
    12 & $"$rodilla$"$ & $"$paella$"$ & Sustitución de 'r' $\rightarrow$ 'p', 'o' $\rightarrow$ 'a', 'd' $\rightarrow$ 'e'. Eliminación de 'i' & 3 * Costo sustitución + 1 * Costo Eliminación\\
    13 & $"$amanda$"$ & $"$ada$"$ & Eliminación de 'm', 'a', 'n'. & 3 * Costo eliminación \\
    \hline
    \end{tabular}
    \caption{Algunos casos de prueba para demostrar operaciones y costo esperado.}
\end{table}
    

\subsubsection{Explicación de Ejecución}

Cada caso ha sido diseñado para cubrir un escenario específico de edición, como inserciones, eliminaciones, sustituciones y transposiciones. A continuación se presenta la justificación por tipo de caso:

\begin{itemize}
    \item \textbf{Igualdad Directa (Casos 1 y 4)}: No requieren cambios, ya que ambas cadenas son idénticas.
    \item \textbf{Transposiciones Necesarias (Casos 2, 3, 7)}: Requieren una transposición para igualar \( S1 \) y \( S2 \), minimizando las operaciones de edición.
    \item \textbf{Inserciones y Eliminaciones (Casos 5, 8, 9, 13)}: Estos casos requieren insertar o eliminar caracteres adicionales en \( S1 \) o \( S2 \) para hacerlas equivalentes.
    \item \textbf{Sustituciones (Casos 6, 11, 12)}: Exigen reemplazar uno o varios caracteres para obtener la cadena deseada.
\end{itemize}

Es importante tener en cuenta que esto es solo una predicción de lo que debería ocurrir, pero pueden existir variaciones. Esto se debe a que la implementación está orientada a encontrar la distancia de edición con costos variables que se generan de manera aleatoria. Por lo tanto, la selección de las operaciones dependerá directamente de los valores con los que se generen los archivos de costos.



% Autor
% Luis Zegarra Stuardo
% 202073628-6
% Tarea 2 y 3
% Algoritmos y Complejidad 2024-2