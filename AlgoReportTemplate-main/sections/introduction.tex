\begin{mdframed}
    \textbf{La extensión máxima para esta sección es de 2 páginas.}
\end{mdframed}

La introducción de este tipo de informes o reportes, tiene como objetivo principal \textbf{contextualizar el problema que se va a analizar}, proporcionando al lector la información necesaria para entender la relevancia del mismo. 

Es fundamental que en esta sección se presenten los antecedentes del problema, destacando investigaciones previas o principios teóricos que sirvan como base para los análisis posteriores. Además, deben explicarse los objetivos del informe, que pueden incluir la evaluación de un algoritmo, la comparación de métodos o la validación de resultados experimentales.

Aunque la estructura y el enfoque siguen principios de trabajos académicos, se debe recordar que estos informes no son publicaciones científicas formales, sino trabajos de pregrado. Por lo tanto, se busca un enfoque claro y directo, que permita al lector comprender la naturaleza del problema y los objetivos del análisis, sin entrar en detalles excesivos. 


Introduction Checklist de \citetitle{GoodScientificPaper} \cite{GoodScientificPaper}, adaptada a nuestro contexto:

\begin{itemize}
\item Indique el \textbf{campo del trabajo} (Análisis y Diseño de algoritmos en Ciencias de la Computación), por qué este campo es importante y qué se ha hecho ya en este área, con las \textbf{citas} adecuadas de la literatura académica o fuentes relevantes.
\item Identifique una \textbf{brecha} en el conocimiento, un desafío práctico, o plantee una \textbf{pregunta} relacionada con la eficiencia, complejidad o aplicabilidad de un algoritmo particular.
\item Resuma el propósito del informe e introduzca el análisis o experimento, dejando claro qué se está investigando o comparando, e indique \textbf{qué es novedoso} o por qué es significativo en el contexto de un curso de pregrado.
\item Evite; repetir el resumen; proporcionar información innecesaria o fuera del alcance de la materia (limítese al análisis de algoritmos o conceptos de complejidad); exagerar la importancia del trabajo (recuerde que se trata de un informe de pregrado); afirmar novedad sin una comparación adecuada con lo enseñado en clase o la bibliografía recomendada.
\end{itemize}



\begin{mdframed}
Recuerde que este es su trabajo, y sólo usted puede expresar con precisión lo que ha aprendido y quiere transmitir. Si lo hace bien, su introducción será más significativa y valiosa que cualquier texto automatizado. ¡Confíe en sus habilidades, y verá que puede hacer un mejor trabajo que cualquier herramienta que automatiza la generación de texto!
\end{mdframed}

---
 
Hoy en día, en la era de la tecnología, la \textbf{calidad y eficiencia de los programas computacionales} están creciendo enormemente. Con el avance de la tecnología, el aumento de la cantidad de datos y la complejidad de los procesos, es fundamental contar con métodos para \textbf{consultar, organizar y manejar} toda esta información de manera eficiente. En este contexto, el campo de \textbf{Análisis y Diseño de Algoritmos en Ciencias de la Computación} cobra especial relevancia, ya que permite abordar la \textbf{resolución de problemas complejos y la creación de programas optimizados}, que pueden ejecutarse en tiempos razonables y con un consumo de memoria reducido.

Un problema recurrente en este campo es el cálculo de la \textbf{distancia mínima de edición}, ampliamente utilizado en aplicaciones como el \textbf{procesamiento de lenguaje natural, la recuperación de información, la biología computacional y la inteligencia artificial}. En estos casos, la necesidad de comparar y procesar cadenas de texto de manera precisa y eficiente es clave para obtener resultados de calidad. El cálculo de la distancia mínima de edición, además de ser una tarea común, es esencial en estos ámbitos, ya que permite medir la similitud entre dos secuencias mediante la transformación de una en otra.

En este documento se \textbf{presentará la implementación de dos algoritmos para calcular la distancia mínima de edición entre dos cadenas}, aplicando dos enfoques: \textbf{fuerza bruta} y \textbf{programación dinámica}. Estos algoritmos permitirán comparar la eficiencia de los enfoques mencionados. Ambos algoritmos incorporan operaciones de \textbf{inserción, eliminación, sustitución y transposición} con costos variables, lo cual aumenta la complejidad y utilidad del cálculo.

El algoritmo de \textbf{fuerza bruta} explora todas las posibles transformaciones de manera exhaustiva, y aunque es un enfoque simple, su tiempo de ejecución crece exponencialmente conforme aumenta el tamaño de las cadenas. Esto se traduce en una solución viable solo para casos de pequeña escala. Por otro lado, el algoritmo de \textbf{programación dinámica} optimiza la búsqueda de soluciones almacenando los resultados de subproblemas ya resueltos, lo cual reduce la cantidad de operaciones necesarias para obtener el resultado final. Sin embargo, esta optimización viene con un mayor consumo de memoria, dado que es necesario almacenar los subproblemas en una estructura de datos.

Este estudio se realiza para \textbf{verificar la eficiencia de ambos métodos}, por lo que se analizarán el \textbf{tiempo de ejecución} y el \textbf{uso de memoria} en diferentes pares de palabras con diversas características. Se dejará evidencia de las implementaciones realizadas, demostrando de manera gráfica los pros y contras de ambos algoritmos. De esta manera, se busca ayudar al lector a comprender las características de cada algoritmo en función del input que recibe, así como la cantidad de operaciones necesarias para lograr obtener una solución.

