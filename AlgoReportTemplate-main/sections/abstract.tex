Este estudio compara dos algoritmos, Fuerza Bruta y Programación Dinámica, utilizando como referencia el algoritmo de Distancia de Levenshtein, para calcular la distancia mínima de edición entre dos cadenas de texto, con costos variables en operaciones como inserción, eliminación, sustitución y transposición. Ambos algoritmos fueron implementados en C++ y se realizaron pruebas con cadenas de diferentes tamaños. Los resultados muestran que la Programación Dinámica es más eficiente en tiempo de ejecución que la Fuerza Bruta, aunque los datos sobre el uso de memoria fueron inconsistentes debido a problemas en la medición. En conclusión, el algoritmo de Programación Dinámica demuestra ser más efectivo, pero la implementación de la medición de memoria y el registro de operaciones requiere mejoras para futuros estudios.



% Autor
% Luis Zegarra Stuardo
% 202073628-6
% Tarea 2 y 3
% Algoritmos y Complejidad 2024-2