\subsubsection{Descripción de la Solución Recursiva}
En programación dinámica, el problema se descompone en subproblemas, almacenando cada resultado dentro de una matriz a medida que se revisan los strings, para evitar cálculos repetitivos. Esto permite calcular la distancia de edición entre \( S1 \) y \( S2 \) de manera más eficiente.


\subsubsection{Relación de Recurrencia}
La relación de recurrencia es la siguiente:
\[
\text{DP}[i][j] = \min \begin{cases} 
    \text{DP}[i-1][j] + \text{costo\_eliminación} \\
    \text{DP}[i][j-1] + \text{costo\_inserción} \\
    \text{DP}[i-1][j-1] + \text{costo\_sustitución} & \text{si } S1[i] \neq S2[j] \\
    \text{DP}[i-2][j-2] + \text{costo\_transposición} & \text{si hay transposición} \\
\end{cases}
\]

Los casos base son:
\[
\text{DP}[0][j] = j \times \text{costo\_inserción} \quad \text{y} \quad \text{DP}[i][0] = i \times \text{costo\_eliminación}
\]

\subsubsection{Identificación de Subproblemas}
Cada subproblema \( \text{DP}[i][j] \) representa la distancia mínima de edición para transformar los primeros \( i \) caracteres de \( S1 \) en los primeros \( j \) caracteres de \( S2 \).



\begin{algorithm}[H]
    \SetKwProg{myproc}{Procedure}{}{}
    \SetKwFunction{AlgorithmName}{distanciaEdicionProgDinamica}
    
    \DontPrintSemicolon
    \footnotesize

    \myproc{\AlgorithmName{S1, S2, operaciones}}{
        m $\leftarrow$ longitud de S1\;
        n $\leftarrow$ longitud de S2\;
        Crear matriz dp de dimensiones $(m + 1) \times (n + 1)$ inicializada en 0\;

        \For{i desde 1 hasta m}{
            dp[i][0] $\leftarrow$ dp[i - 1][0] + costo\_delete(S1[i - 1])\;
            Escribir $"$eliminación$"$ + S1[i - 1] en operaciones\;
        }

        \For{j desde 1 hasta n}{
            dp[0][j] $\leftarrow$ dp[0][j - 1] + costo\_insert(S2[j - 1])\;
            Escribir $"$inserción$"$ + S2[j - 1] en operaciones\;
        }

        \For{i desde 1 hasta m}{
            \For{j desde 1 hasta n}{
                costoSustitucion $\leftarrow$ dp[i - 1][j - 1] + costo\_sub(S1[i - 1], S2[j - 1])\;
                costoInsercion $\leftarrow$ dp[i][j - 1] + costo\_insert(S2[j - 1])\;
                costoEliminacion $\leftarrow$ dp[i - 1][j] + costo\_delete(S1[i - 1])\;
                
                dp[i][j] $\leftarrow$ mínimo entre \{costoSustitucion, costoInsercion, costoEliminacion\}\;

                \uIf{dp[i][j] es costoSustitucion}{
                    Escribir $"$sustitución$"$ + S1[i - 1] + $"$->$"$ + S2[j - 1] en operaciones\;
                }
                \uElseIf{dp[i][j] es costoInsercion}{
                    Escribir $"$inserción$"$ + S2[j - 1] en operaciones\;
                }
                \ElseIf{dp[i][j] es costoEliminacion}{
                    Escribir $"$eliminación$"$ + S1[i - 1] en operaciones\;
                }

                \If{i $>$ 1 y j $>$ 1 y S1[i - 1] $=$ S2[j - 2] y S1[i - 2] $=$ S2[j - 1]}{
                    costoTransposicion $\leftarrow$ dp[i - 2][j - 2] + costo\_transpose(S1[i - 2], S1[i - 1])\;
                    \If{dp[i][j] $>$ costoTransposicion}{
                        dp[i][j] $\leftarrow$ costoTransposicion\;
                        Escribir $"$transposición$"$ + S1[i - 2] + S1[i - 1] en operaciones\;
                    }
                }
            }
        }

        \Return dp[m][n]\;
    }
    \caption{Algoritmo de distancia de edición con programación dinámica.}
    \label{alg:distanciaEdicionProgDinamica}
\end{algorithm}

\subsubsection{Ejemplo Paso a Paso}

La función \texttt{distanciaEdicionProgDinamica} calcula la distancia mínima de edición entre dos cadenas, \( S1 \) y \( S2 \), mediante un enfoque de programación dinámica, usando una matriz de costos (\texttt{dp}) para almacenar soluciones de subproblemas. A continuación, se presenta un resumen de los pasos clave:

\begin{itemize}
    \item \textbf{Inicialización de Casos Base:}
    \begin{itemize}
        \item Rellena la primera columna de \texttt{dp} para reflejar el costo de eliminar cada carácter en \( S1 \) cuando \( S2 \) está vacío.
        \item Rellena la primera fila de \texttt{dp} para reflejar el costo de insertar cada carácter de \( S2 \) cuando \( S1 \) está vacío.
    \end{itemize}
    
    \item \textbf{Cálculo de Costos de Operaciones:}
    \begin{itemize}
        \item Para cada posición \( (i, j) \) en la matriz, calcula:
        \begin{itemize}
            \item \textbf{Sustitución:} El costo de reemplazar \( S1[i-1] \) por \( S2[j-1] \).
            \item \textbf{Inserción:} El costo de insertar \( S2[j-1] \).
            \item \textbf{Eliminación:} El costo de eliminar \( S1[i-1] \).
        \end{itemize}
    \end{itemize}
    
    \item \textbf{Registro de Operaciones:} 
    \begin{itemize}
        \item Se escribe en el archivo de operaciones el tipo de operación seleccionada en cada paso (sustitución, inserción, o eliminación) basada en el menor costo.
    \end{itemize}

    \item \textbf{Verificación de Transposición (Opcional):} 
    \begin{itemize}
        \item Si \( S1[i-1] \) y \( S1[i-2] \) están invertidos respecto a \( S2[j-1] \) y \( S2[j-2] \), calcula el costo de transponerlos y actualiza \texttt{dp[i][j]} si este costo es menor que los otros.
        \item Registra la operación de transposición en el archivo de operaciones si es elegida.
    \end{itemize}

    \item \textbf{Resultado:} Al finalizar, \texttt{dp[m][n]} contiene el costo mínimo para transformar \( S1 \) en \( S2 \).
\end{itemize}

\subsubsection{Análisis de Complejidad}
La complejidad temporal de la versión dinámica es \( O(m \times n) \), ya que se calcula cada celda de la matriz \( dp \) una sola vez. La complejidad espacial también es \( O(m \times n) \) debido a la matriz utilizada para almacenar los subproblemas. La inclusión de transposiciones y costos variables agrega un costo adicional en el cálculo de cada celda de \( dp \), pero no cambia la complejidad asintótica.



% Autor
% Luis Zegarra Stuardo
% 202073628-6
% Tarea 2 y 3
% Algoritmos y Complejidad 2024-2