\begin{algorithm}[H]
    \SetKwProg{myproc}{Procedure}{}{}
    \SetKwFunction{AlgorithmName}{distanciaEdicionFuerzaBruta}
    
    \DontPrintSemicolon
    \footnotesize

    \myproc{\AlgorithmName{S1, S2, i, j, operaciones}}{
        \uIf{i es igual a la longitud de S1}{
            costo $\leftarrow$ 0\;
            \For{k desde j hasta la longitud de S2 - 1}{
                Escribir $"$inserción$"$ + S2[k] en operaciones\;
                costo $\leftarrow$ costo + costo\_insert(S2[k])\;
            }
            \Return costo\;
        }
        \uElseIf{j es igual a la longitud de S2}{
            costo $\leftarrow$ 0\;
            \For{k desde i hasta la longitud de S1 - 1}{
                Escribir $"$eliminación$"$ + S1[k] en operaciones\;
                costo $\leftarrow$ costo + costo\_delete(S1[k])\;
            }
            \Return costo\;
        }
        
        costoSustitucion $\leftarrow$ costo\_sub(S1[i], S2[j]) + \AlgorithmName{S1, S2, i + 1, j + 1, operaciones}\;
        costoInsercion $\leftarrow$ costo\_insert(S2[j]) + \AlgorithmName{S1, S2, i, j + 1, operaciones}\;
        costoEliminacion $\leftarrow$ costo\_delete(S1[i]) + \AlgorithmName{S1, S2, i + 1, j, operaciones}\;

        costoTransposicion $\leftarrow$ INT\_MAX\;
        \If{i + 1 $<$ longitud de S1 y j + 1 $<$ longitud de S2 y S1[i] $=$ S2[j + 1] y S1[i + 1] $=$ S2[j]}{
            costoTransposicion $\leftarrow$ costo\_transpose(S1[i], S1[i + 1]) + \AlgorithmName{S1, S2, i + 2, j + 2, operaciones}\;
        }

        costoMinimo $\leftarrow$ mínimo entre \{costoSustitucion, costoInsercion, costoEliminacion, costoTransposicion\}\;

        \uIf{costoMinimo es igual a costoSustitucion}{
            Escribir $"$sustitución$"$ + S1[i] + $"$->$"$ + S2[j] en operaciones\;
        }
        \uElseIf{costoMinimo es igual a costoInsercion}{
            Escribir $"$inserción$"$ + S2[j] en operaciones\;
        }
        \ElseIf{costoMinimo es igual a costoEliminacion}{
            Escribir $"$eliminación$"$ + S1[i] en operaciones\;
        }
        \ElseIf{costoMinimo es igual a costoTransposicion}{
            Escribir $"$transposición$"$ + S1[i] + S1[i + 1] en operaciones\;
        }

        \Return costoMinimo\;
    }
    \caption{Algoritmo de distancia de edición con fuerza bruta.}
    \label{alg:distanciaEdicionFuerzaBruta}
\end{algorithm}

\subsubsection{Ejemplo Paso a Paso}

La función \texttt{distanciaEdicionFuerzaBruta} calcula la distancia mínima de edición entre dos cadenas, \( S1 \) y \( S2 \), usando un enfoque de fuerza bruta. A continuación se describen los pasos clave:

\begin{itemize}
    \item \textbf{Condiciones de Fin:} Si se alcanza el final de \( S1 \) (índice \( i \)), inserta todos los caracteres restantes de \( S2 \) y suma sus costos. Si se alcanza el final de \( S2 \) (índice \( j \)), elimina los caracteres restantes de \( S1 \) y acumula el costo de cada eliminación.

    \item \textbf{Cálculo de Costos de Operaciones:} 
    \begin{itemize}
        \item \textbf{Sustitución:} Calcula el costo de reemplazar \( S1[i] \) por \( S2[j] \), avanzando ambos índices.
        \item \textbf{Inserción:} Calcula el costo de insertar \( S2[j] \) en \( S1 \), avanzando solo \( j \).
        \item \textbf{Eliminación:} Calcula el costo de eliminar \( S1[i] \), avanzando solo \( i \).
    \end{itemize}

    \item \textbf{Costo de Transposición:} Si los caracteres \( S1[i] \) y \( S1[i+1] \) están invertidos con respecto a \( S2[j] \) y \( S2[j+1] \), calcula el costo de transponer estos caracteres y avanza dos posiciones en ambas cadenas.

    \item \textbf{Selección y Registro de Operación de Menor Costo:} Una vez calculados los costos, se identifica el mínimo y se registra únicamente la operación correspondiente en un archivo de salida.

    \item \textbf{Retorno del Costo Mínimo:} Devuelve el menor costo calculado, representando el costo total mínimo para transformar \( S1 \) en \( S2 \) desde las posiciones actuales.
\end{itemize}


\subsubsection{Análisis de Complejidad}
La complejidad temporal del algoritmo de fuerza bruta es exponencial, \( O(4^{\max(m, n)}) \), donde \( m \) y \( n \) son las longitudes de \( S1 \) y \( S2 \), respectivamente. La falta de almacenamiento de subproblemas resueltos obliga a recalcular cada transformación posible, resultando en una alta demanda de tiempo de ejecución para cadenas largas. La complejidad espacial es \( O(\max(m, n)) \) debido a la profundidad máxima de la pila de recursión.

La inclusión de transposiciones y costos variables aumenta la cantidad de posibles operaciones a evaluar en cada paso, incrementando la carga computacional en comparación con la versión estándar del algoritmo de distancia de edición.


% Autor
% Luis Zegarra Stuardo
% 202073628-6
% Tarea 2 y 3
% Algoritmos y Complejidad 2024-2