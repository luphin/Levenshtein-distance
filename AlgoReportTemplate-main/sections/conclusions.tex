Los resultados obtenidos a partir de las simulaciones demuestran de manera clara la superioridad del algoritmo de Programación Dinámica frente al de Fuerza Bruta en cuanto al tiempo de ejecución. A medida que el tamaño de las cadenas de entrada aumenta, el algoritmo de Programación Dinámica muestra una eficiencia notablemente mayor en términos de tiempo, lo que valida su eficacia para resolver el problema planteado.

En cuanto al análisis de la memoria, los resultados no son concluyentes debido a la deficiencia en la implementación de la medición de memoria. Las variaciones observadas en el uso de memoria pueden atribuirse tanto a limitaciones del compilador, que optimiza el uso de memoria en ciertos casos, como a posibles fallos en la implementación de los registros dentro del archivo \texttt{operaciones.txt}, lo cual limitó la revisión sobre los pasos que realizaron cada algoritmo para encontrar la solución, ya que, se registraron gran parte de las revisiones realizadas y no solos las que aportaban en el cálculo de la solución final.

A pesar de estas dificultades, el análisis realizado confirma que ambos algoritmos son adecuados para resolver el problema de calcular la menor distancia de edición con costos variables, aunque la obtención de los pasos intermedios de cada algoritmo no fue posible debido a los problemas con la implementación del manejo de registros. Estos resultados contribuyen a una mejor comprensión de las ventajas y limitaciones de los algoritmos en el contexto de la distancia de edición con costos variables, destacando la importancia de una implementación precisa para obtener métricas confiables.



% Autor
% Luis Zegarra Stuardo
% 202073628-6
% Tarea 2 y 3
% Algoritmos y Complejidad 2024-2